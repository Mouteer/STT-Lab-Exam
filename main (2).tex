\documentclass[12pt]{article}
\usepackage{fancyhdr}
\usepackage{geometry}
% Set up page geometry
\geometry{a4paper, margin=1in}
% Header and Footer settings
\pagestyle{fancy}
\fancyhf{} % Clear all header and footer fields
\fancyhead[C]{Mysteries of Science and Discovery} % Header center
\fancyfoot[L]{Maulana Abul Kalam Azade University of Technology} % Footer left
\fancyfoot[R]{\thepage} % Footer right (page number)

\begin{document}

% Restore fancy page style after the title page
\thispagestyle{fancy}
\pagestyle{fancy}

% Section 1
\section{The Quantum Realm}
In the infinitesimal world of quantum mechanics, particles dance in ways that defy classical physics. Electrons seem to exist in multiple places simultaneously, while photons exhibit both wave-like and particle-like properties. The uncertainty principle governs this realm, where the very act of observation can alter the outcome of an experiment. Scientists continue to grapple with the implications of quantum entanglement, a phenomenon Einstein famously referred to as "spooky action at a distance."

% Section 2
\section{The Enigma of Dark Matter}
Invisible to the naked eye and undetectable by conventional means, dark matter remains one of the greatest mysteries in modern astrophysics. Its presence is inferred from gravitational effects on visible matter, unexplained galactic rotations, and the large-scale structure of the universe. Despite comprising an estimated 85% of all matter in the cosmos, its nature eludes researchers. Theories abound, from weakly interacting massive particles (WIMPs) to axions, but concrete evidence remains tantalizingly out of reach.

\end{document}
